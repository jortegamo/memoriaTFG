%%%%%%%%%%%%%%%%%%%%%%%%%%%%%%%%%%%%%%%%%%%%%%%%%%%%%%%%%%
%%%%%%%                  															       %%%%%%
%%%%%%%                               CAPITULO 4: DESARROLLO DE LA APLICACI�N                                %%%%%%
%%%%%%%  																	       %%%%%%
%%%%%%%%%%%%%%%%%%%%%%%%%%%%%%%%%%%%%%%%%%%%%%%%%%%%%%%%%%



\chapter{Desarrollo de la aplicaci�n}\label{cap:desarrollo}
\paragraph{}
El desarrollo de cualquier aplicaci�n viene despu�s de su dise�o. No necesariamente es el paso final, puesto que esta etapa sirve de feedback en el proceso general y muchas veces hay que retomar el dise�o para establecer nuevos requisitos o cambiar los actuales.
\section{Aprendizaje}
\paragraph{}
Una vez dise�ada la aplicaci�n y establecidas las herramientas a utilizar, el siguiente paso es profundizar en cada una de ellas. El aprendizaje de nuevas herramientas es lento pero el afianzamiento de los conocimientos necesarios de manera correcta nos evitar� muchos problemas en el futuro. Por esto hay que esforzarse en comprenderlos y la primera fase del Desarrollo de la aplicaci�n es el aprendizaje.



\subsection{CSS3 y preprocesadores}
Aqu� explicar� Qu� he aprendido de css3 y de los preprocesadores less y saas.
\subsection{API SoundCloud}
Aqu� explicar� el servicio de almacenamiento de SoundCloud.
\section{Componiendo el escenario}
Peque�a introducci�n
\subsection{Entorno de desarrollo}
WebStorm. Foto de la interfaz.
\subsection{Paquetes}
Lista de paquetes necesarios y sus funcionalidades
\subsection{Collecciones}
Explicar c�mo se han desarrollado las entidades en colecciones y la estructura de los objetos guardados. C�mo se relacionan.
\subsection{Enrutamiento}
Explicar Iron Router y su potencial.
\subsection{Mixins iniciales}
Foto de los mixines iniciales y que son necesarios.
\section{Layout Principal}
Estructura general de la vista de la aplicaci�n.
\section{Registro de usuarios}
Peque�a introduccion hablar de los servicios de meteor autom�ticos para sign.
\subsection{Sign In y Sign Up}
Funcionalidad y pantallas
\subsection{Verificaci�n de Email}
Funcionalidad y pantallas
\subsection{Cambio y recuperaci�n de contrase�a}
Funcionalidad y pantallas
\subsection{Servicios agregados}
Funcionalidad y pantallas.

\section{Sidebar}
Explicar que contenido se muestra en sidebar y que es el enlace a todos los recursos de la aplicaci�n desde cualquier pantalla.
\section{Formularios}
Peque�a introducci�n enumerar los formularios creados.
\subsection{Dise�o de plantilla din�mica}
Explicar c�mo se crea una plantilla din�mica para los formularios.
\subsection{Creadores de entidades}
Enumerarlos y pantallas.
\subsection{Conversaciones}
Hablar del componente inputMembers.
\subsection{Grabaciones}
Hablar del grabador. (extenso).

\section{Entidades}
Hay que dar forma a las entidades visualmente.
\subsection{Canales}
Hablar de los canales.
\subsection{Lecciones}
Hablar de las lecciones.
\subsection{Secciones}
Hablar de las secciones.
\subsection{Grabaciones}
Hablar de la reproducci�n de las grabaciones. (extenso)
\subsection{Listas de reproducci�n}
Asociadas a una secci�n.
\subsection{Conversaciones}
Hablar de las conversaciones y de los roles.
\section{Perfil de usuario}
\subsection{Rol due�o}
\subsection{Rol visitante}
\subsection{Contactos}
\section{Tutoriales}
\section{Landing}
\section{BrowseCrossing}
\section{Full Responsive}
\section{Despliegue}