\chapter{Pruebas de Validaci�n}
En este cap�tulo se representa la fase de experimentaci�n del proyecto. Esta fase se basa en la realizaci�n de una prueba en la que se ha medido la utilidad de la aplicaci�n y su rendimiento en un entorno real.
\section{Motivaci�n}
La principal motivaci�n de este experimento es analizar el comportamiento de la aplicaci�n en la realidad. Dicho comportamiento deber� cumplir estrictamente los requisitos propuestos en la secci�n \ref{sec:requisitos}.
\section{Planteamiento y objetivos}
Una vez desplegada la aplicaci�n en Heroku \cite{baz14} y realizadas las pruebas globales oportunas se ha procedido a generar contenido en la misma y a plantear el experimento. 
\paragraph{}
El experimento consistir� en hacer accesible la aplicaci�n a un grupo de alumnos mediante la difusi�n de la url donde ha sido alojada. Dichos alumnos, siguiendo la gu�a de uso elaborada para esta fase del proyecto, explotar�n todas las caracter�sticas de la aplicaci�n y su funcionalidad. Por otra parte se ha integrado a la aplicaci�n el servicio de Google Analytics \footnote{\url{https://analytics.google.com}} para controlar y analizar el flujo de usuarios dentro de la aplicaci�n.
\paragraph{}
Los objetivos perseguidos se han resumido en las siguientes caracter�sticas:
\begin{itemize}
	\item \textbf{Funcional:} la aplicaci�n debe cumplir todos los requisitos establecidos.
	\item \textbf{Atractiva e intuitiva:} que los alumnos aprecien el atractivo de las interfaces y el flujo de la aplicaci�n.
	\item \textbf{Fluida y �ptima:} la aplicaci�n debe comportarse de manera fluida con m�s de un usuario utiliz�ndola.
	\item \textbf{�til:} la aplicaci�n debe suponer una herramienta de trabajo para los alumnos. 
\end{itemize}
\paragraph{}
Para medir y analizar el cumplimiento de los objetivos marcados se ha desarrollado una encuesta o formulario que se ha difundido junto con la gu�a de uso. Los alumnos, una vez completada la gu�a, aportar�n informaci�n sobre su experiencia contestando a las preguntas de dicho formulario.
\section{Proceso y realizaci�n}
El proceso y la realizaci�n ha sido muy sencilla, Se han habilitado los recursos necesarios (gu�a y formulario) de forma remota y se ha procedido al env�o de dichos enlaces a un grupo de alumnos.
\section{Resultados y an�lisis}
Una vez que los alumnos han probado la aplicaci�n y han rellenado la encuesta se ha procedido al an�lisis de los resultados. 