\chapter{Resumen}
El presente trabajo fin de grado se ha realizado la implementaci�n de una aplicaci�n web que sirva de herramienta a desarrolladores de manera que puedan aprender y compartir conocimientos sobre las tecnolog�as de software actuales. El principal objetivo ha sido crear una herramienta de grabaci�n de c�digo y audio sobre un editor de forma que el elemento b�sico de intercambio de informaci�n en la plataforma fuera lo m�s visual e interactivo posible. Para lograr esa interactividad se ha desarrollado un sistema de respuestas a estos elementos en forma de nuevas grabaciones y la posibilidad de comenzar su grabaci�n en cualquier instante de la reproducci�n de los mismos. Para una mejor proyecci�n de la aplicaci�n y una mejor organizaci�n de los contenidos, se han creado espacios denominados canales y lecciones con funciones en los que organizar dichos elementos base. Adem�s se ha creado un m�dulo que permite la comunicaci�n mediante conversaciones y el env�o de emails entre usuarios. Tambi�n se han desarrollado espacios donde se exponen las principales caracter�sticas de la aplicaci�n y documentaci�n en formato de v�deo. 
\paragraph{}
Para un correcto desarrollo han sido estudiadas y aplicadas tecnolog�as como WebRTC, MeteorJS, Bootstrap, FlexBox, Javascript, JQuery, SASS, AceEditor, SoundCloud API, MongoDB y HTML5.
\paragraph{}
En este proyecto se ha seguido una metodolog�a realimentada. El desarrollo de la aplicaci�n se ha dividido en la implementaci�n de distintos prototipos cuyo desarrollo ha servido de base para el siguiente. En cada prototipo se han extra�do requisitos asociados en mayor o menor medida a los principales objetivos propuestos.
\paragraph{}
Una vez finalizada la implementaci�n de la aplicaci�n y realizado su despliegue mediante el sistema de Hosting de Heroku, se ha procedido con la realizaci�n de una prueba de validaci�n basada en la distribuci�n de forma an�nima de una encuesta a alumnos de 4� grado escogidos al azar y con posibilidades de mostrar inter�s por la herramienta desarrollada. Adem�s se ha realizado un estudio de navegaci�n en el sitio gracias a la tecnolog�a de Google Analytics.
\paragraph{}
El an�lisis de los resultados obtenidos ofrece una valoraci�n positiva sobre el flujo de navegaci�n, la funcionalidad, el atractivo, la usabilidad y la acogida del producto conseguido.

